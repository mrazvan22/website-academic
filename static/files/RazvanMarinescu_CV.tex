%%%%%%%%%%%%%%%%%%%%%%%%%%%%%%%%%%%%%%%%%
% Plasmati Graduate CV
% LaTeX Template
% Version 1.0 (24/3/13)
%
% This template has been downloaded from:
% http://www.LaTeXTemplates.com
%
% Original author:
% Alessandro Plasmati (alessandro.plasmati@gmail.com)
%
% License:
% CC BY-NC-SA 3.0 (http://creativecommons.org/licenses/by-nc-sa/3.0/)
%
% Important note:
% This template needs to be compiled with XeLaTeX.
% The main document font is called Fontin and can be downloaded for free
% from here: http://www.exljbris.com/fontin.html
%
%%%%%%%%%%%%%%%%%%%%%%%%%%%%%%%%%%%%%%%%%

%-------------------------------------------------------------
%	PACKAGES AND OTHER DOCUMENT CONFIGURATIONS
%-------------------------------------------------------------

\documentclass[a4paper,10pt]{article} % Default font size and paper size

%\usepackage{fontspec} % For loading fonts
%\defaultfontfeatures{Mapping=tex-text}
%\setmainfont[SmallCapsFont = Fontin SmallCaps]{Fontin} % Main document font

%\usepackage{xunicode,xltxtra,url,parskip} % Formatting packages

\usepackage[usenames,dvipsnames]{xcolor} % Required for specifying custom colors

\usepackage[
    top    = 1.4cm,
    bottom = 1.4cm,
    left   = 1.4cm,
    right  = 1.4cm]{geometry}

%\usepackage{fullpage} % Margin formatting of the A4 page, an alternative to
%layaureo can be \usepackage{fullpage}
% To reduce the height of the top margin uncomment:
%\addtolength{\voffset}{-1.3cm}
%\addtolength{\hoffset}{-1.3cm}

\usepackage{hyperref} % Required for adding links	and customizing them
\definecolor{linkcolour}{rgb}{0,0.2,0.6} % Link color
\hypersetup{colorlinks,breaklinks,urlcolor=linkcolour,linkcolor=linkcolour} %
% Set link colors throughout the document

%\usepackage{titlesec} % Used to customize the \section command
%\titleformat{\section}{\Large\scshape\raggedright}{}{0em}{}[\titlerule]
% Text formatting of sections

%\titlespacing{\section}{0pt}{3pt}{3pt} % Spacing around sections

\begin{document}

\pagestyle{empty} % Removes page numbering

%\font\fb=''[cmr10]''
% Change the font of the \LaTeX command under the skillssection

%-------------------------------------------------------------------
%	NAME AND CONTACT INFORMATION
%-------------------------------------------------------------------

% Your name
\par{\centering{\huge Razvan Valentin Marinescu}\bigskip\par}

\begin{tabular}{ll}
 Website: & \url{http://razvan.csail.mit.edu}\\
 Github: & \url{https://github.com/mrazvan22}\\
 Twitter: & \url{https://twitter.com/RazMarinescu}\\
 Address: & MIT CSAIL, 32 Vassar St., Office D475B\\
 Email: & razvan [at] csail.mit.edu\\

\end{tabular}


%-------------------------------------------------------------------------------

%	EDUCATION
%-------------------------------------------------------------------------------

\section*{Research Interests}
Machine Learning, Medical Applications, Computer Vision, Bayesian Statistics, Inference

\section*{Education}

\begin{tabular}{r|p{15.7cm}}
2014 & \large\textbf{4-Year PhD in Medical Imaging, University College London}\\
- 2018 & PhD project: "Disease Progression Modelling and Evaluation in Alzheimer's Disease and Posterior Cortical Atrophy"\\
& \small Supervisors: Prof. Daniel Alexander, Dr. Sebastian Crutch, Dr. Neil Oxtoby\\
% & I developed statistical and computational models of neurodegenerative disease progression. Applied them to different types of dementia such as typical Alzheimer's disease and Posterior Cortical Atrophy and in order to perform probabilistic disease staging and forecast the future evolution of subjects. I also organised the international TADPOLE Competition, which aims to identify algorithms and features that best predict the evolution of subjects at-risk of Alzheimer's disease.\\\\

& Research focus: bayesian latent-variable models, machine learning, neuroimaging, disease progression modelling.\\

\\
2010 & \large\textbf{4-Year MEng in Computer Science, Imperial College London}\\
- 2014 & \emph{First Class Honours} (top 10\% of class in final year) \\
& Master thesis: ``On a new metric to compare internal structures in biological
networks''\\
& \small Supervisor: Dr. Natasa Przulj\\
& Research focus: graph analysis, applications to biological and economic networks
% & I developed a novel graph signature that was capturing the network structure around a particular node. I applied the signature to networks representing world-trade, protein-protein interactions and matabolic pathways and showed that it correlates with different network properties, such as oil price (trade network), key metabolic pathways (metabolic network) and key protein functions (protein network). \\
\end{tabular}


%-------------------------------------------------------------------------------
%	WORK EXPERIENCE
%-------------------------------------------------------------------------------

\section*{Employment}



\begin{tabular}{r|p{15cm}}

Jan 2019 & \large\textbf{Postdoctoral Associate at Massachusetts Institute of Technology} \\
- present & \emph{Advisor: Pollina Golland}\\
& Research focus: brain image analysis, classifier interpretability, generative modelling\\
\multicolumn{2}{c}{} \\

%------------------------------------------------


Jan 2016 & \large\textbf{Teaching Assistant in Computational Modelling, UCL} \\
- Apr 2018 & Taught computational modelling, bayesian statistics and numerical optimisation to Master students. Marked the students' coursework.\\
\multicolumn{2}{c}{} \\

%------------------------------------------------


Sep 2014 & \large\textbf{Student Residence Advisor, University College London} \\
- Aug 2018 & Provided pastoral support to students and emergency support.\\
\multicolumn{2}{c}{} \\

%------------------------------------------------


Oct 2012 & \large\textbf{Teaching Assistant in Programming, Imperial College London} \\
- Dec 2013 & Taught Haskell, Java and C to undergraduate students. Weekly marking of students' coursework.\\
\multicolumn{2}{c}{} \\


 %------------------------------------------------

Mar - Sep & \large\textbf{Industrial Placement at J.P. Morgan Chase \& Co, Emerging
Markets}\\
2013 & \emph{Assisted the retirement of a legacy system that was processing end-of-day market risk.}.\\
\multicolumn{2}{c}{} \\
%------------------------------------------------

Jul - Sep & \large\textbf{Summer Internship at Goldman Sachs, Equities Technology}\\
2012 & \emph{Built programmes that automatically re-factored the Java source-code of a trading system. Learned about financial nstruments and live market data.}.\\


%------------------------------------------------


\multicolumn{2}{c}{} \\
\end{tabular}
\vspace{-3em}

%-------------------------------------------------------------------------------
%	Awards
%-------------------------------------------------------------------------------

\section*{Awards}

\begin{tabular}{r|p{15.7cm}}
2017 & Runner up (jointly) for the Francois Erbsmann Prize at the IPMI conference.\\
2015-17 & Travel and registration fellowships for several conferences: IPMI, AAIC and Human Brain Project.\\
2013 & DAAD Scholarship for doing a German Language course in Aachen, Germany over the summer.\\
%& \\
2011 & Prize for the best undergraduate project in Artificial Intelligence, Imperial College London\\
%& \\
2010 & Sponsored visit to Brussels, at the NATO Headquarters, for the
achievements in international projects and Olympiads.\\
%& \\
2009 & Grand Prize at the International Space Settlement Design Competition offered by NASA Johnsons Space Center.\\
%& \\
2008 & Diploma of Excellency awarded by the Government of Romania for "impressive problem-solving skills".\\
%& \\
2007 & Bronze Medal ar the 6th International Computer Project Competition
"Informatix".\\
%& \\
& Silver Medal at the \emph{National Mathematics Olympiad} in Romania.
\end{tabular}

\section*{Other significant activities}
\begin{tabular}{r|p{15cm}}
2019-20 & President of the MIT Postdoctoral Association\\
2016-17 & Taught Robotics and Computer Graphics courses at the Oxford for Romania Summer School\\
2011-14 & Year representative at Imperial College faculty meetings\\
\end{tabular}


%-------------------------------------------------------------------------------
%	skills and competences
%-------------------------------------------------------------------------------

%\section*{Skills and competences}
%
%\begin{tabular}{rl}
%\textbf{Programming:} & C/C++, Java, x86 Assembly, Python, Perl, SQL, Haskell,
%Prolog, {\LaTeX}\\
%& \\
%\textbf{Computer Systems:} & Ubuntu, Windows, Version control systems(Git)\\
%& \\
%\textbf{Languages:} & Romanian (native), German (fluent), French (basic), Spanish (basic)
%\end{tabular}
%
%%-------------------------------------------------------------------------------
%%	INTERESTS AND ACTIVITIES
%%-------------------------------------------------------------------------------
%
%\section*{Interests and Activities}
%\begin{itemize}
%
% \item As a \textbf{Year Representative} in the Department of Computing, I developed leadership
%and speaking skills by putting forward the academic concerns of students in
%council meetings.\\
%
%  \item \textbf{Treasurer} in the Romanian Society, managing the finances and
%organising various events such as bar and cinema nights, football games and food
%nights. \\
%
%  \item \textbf{First Aider} in St John's Ambulance, offering first aid on various
%  events such as the London Summer Ball, Wimbledon Championships and Arsenal games.\\
%
%  \item Swimming, playing guitar and ice skating with friends in my free time. \\
%
%  \item Wikipedia Editing: Wrote in several scientific articles, such as Gesture Recognition, Affective Computing and Hopfield Network. \\
%
%\end{itemize}



 % USE HARVARD style
\section*{First author publications}

\definecolor{darkgreen}{rgb}{1,0.6,0}
\newcommand{\poster}{\textcolor{red}{Poster}}
\newcommand{\talk}{\textcolor{darkgreen}{Talk}}
\newcommand{\journal}{\textcolor{blue}{Journal}}


\begin{itemize}
\subsection*{2019}
% USE HARVARD style
\item[\poster] \textbf{Marinescu, R.V.}, Lorenzi, M., Blumberg, S., Young, A.L., Morell, P.P., Oxtoby, N.P., Eshaghi, A., Yong, K.X., Crutch, S.J. and Alexander, D.C., 2019. Disease Knowledge Transfer across Neurodegenerative Diseases. MICCAI, 2019.
\item[\talk] \textbf{Marinescu, R.V.}, Alexander, D.C. and Golland, P., 2019. BrainPainter: A software for the visualisation of brain structures, biomarkers and associated pathological processes, MICCAI MBIA Workshop, 2019
\item[\talk] \textbf{Marinescu, R.V.}, Oxtoby, N.P., Young, A.L., Bron, E.E., Toga, A.W., Weiner, M.W., Barkhof, F., Fox, N.C., Golland, P., Klein, S. and Alexander, D.C., 2019, October. TADPOLE challenge: Accurate alzheimer’s disease prediction through crowdsourced forecasting of future data. In International Workshop on PRedictive Intelligence In MEdicine (pp. 1-10). Springer, Cham.
\item[\journal] \textbf{Marinescu, R.V.}, Eshaghi, A., Lorenzi, M., Young, A.L., Oxtoby, N.P., Garbarino, S., Crutch, S.J., Alexander, D.C. and Alzheimer's Disease Neuroimaging Initiative, 2019. DIVE: A spatiotemporal progression model of brain pathology in neurodegenerative disorders. NeuroImage, 192, pp.166-177.
\item[\journal] (*joint first-authors) *Firth, N.C., *Primativo, S., *\textbf{Marinescu, R.V.}, Shakespeare, T.J., Suarez-Gonzalez, A., Lehmann, M., Carton, A., Ocal, D., Pavisic, I., Paterson, R.W. and Slattery, C.F., 2019. Longitudinal neuroanatomical and cognitive progression of posterior cortical atrophy. Brain.
\subsection*{2018}
\item[\journal] \textbf{Marinescu, R.V.}, Oxtoby, N.P., Young, A.L., Bron, E.E., Toga, A.W., Weiner, M.W., Barkhof, F., Fox, N.C., Klein, S. and Alexander, D.C., 2018. TADPOLE Challenge: Prediction of Longitudinal Evolution in Alzheimer's Disease. arXiv preprint arXiv:1805.03909.
\subsection*{2017}
\item[\talk] \textbf{Marinescu, R.V.}, Eshaghi, A., Lorenzi, M., Young, A.L., Oxtoby, N.P., Garbarino, S., Shakespeare, T.J., Crutch, S.J., Alexander, D.C. and Alzheimer’s Disease Neuroimaging Initiative, 2017, June. A vertex clustering model for disease progression: application to cortical thickness images. In International Conference on Information Processing in Medical Imaging (pp. 134-145). Springer, Cham.
\item[\poster] \textbf{Marinescu, R.V.}, Primativo, S., Young, A.L., Oxtoby, N.P., Firth, N.C., Eshaghi, A., Garbarino, S., Cardoso, J.M., Yong, K., Fox, N.C. and Lehmann, M., 2017. Analysis Of The Heterogeneity Of Posterior Cortical Atrophy: Data-driven Model Predicts Distinct Atrophy Patterns For Three Different Cognitive Subgroups. Alzheimer's \& Dementia: The Journal of the Alzheimer's Association, 13(7), pp.P106-P108.
\subsection*{2016}
\item[\poster] \textbf{Marinescu, R.V.}, Young, A.L., Oxtoby, N.P., Firth, N.C., Lorenzi, M., Eshaghi, A., Wottschel, V., Cardoso, M.J., Modat, M., Yong, K. and Primativo, S., 2016. A Data-driven Comparison Of The Progression Of Brain Atrophy In Posterior Cortical Atrophy And Alzheimer's Disease. Alzheimer's \& Dementia: The Journal of the Alzheimer's Association, 12(7), pp.P401-P402.
\end{itemize}

% USE HARVARD style
\section*{Joint publications}
\begin{itemize}
\subsection*{2019}
\item[\journal] Eshaghi, A., \textbf{Marinescu, R.V.}, Young, A.L., Firth, N.C., Prados, F., Jorge Cardoso, M., Tur, C., De Angelis, F., Cawley, N., Brownlee, W.J. and De Stefano, N., 2018. Progression of regional grey matter atrophy in multiple sclerosis. Brain, 141(6), pp.1665-1677.
\item[\poster] Slator, P.J., Hutter, J., \textbf{Marinescu, R.V.}, Palombo, M., Young, A.L., Jackson, L.H., Ho, A., Chappell, L.C., Rutherford, M., Hajnal, J.V. and Alexander, D.C., 2019, June. InSpect: INtegrated SPECTral Component Estimation and Mapping for Multi-contrast Microstructural MRI. In International Conference on Information Processing in Medical Imaging (pp. 755-766). Springer, Cham.
\item[\journal] Garbarino, S., Lorenzi, M., Oxtoby, N.P., Vinke, E.J., \textbf{Marinescu, R.V.}, Eshaghi, A., Ikram, M.A., Niessen, W.J., Ciccarelli, O., Barkhof, F. and Schott, J.M., 2019. Differences in topological progression profile among neurodegenerative diseases from imaging data, eLife
\subsection*{2018}
\item[\journal] Young, A.L., \textbf{Marinescu, R.V.}, Oxtoby, N.P., Bocchetta, M., Yong, K., Firth, N.C., Cash, D.M., Thomas, D.L., Dick, K.M., Cardoso, J. and van Swieten, J., 2018. Uncovering the heterogeneity and temporal complexity of neurodegenerative diseases with Subtype and Stage Inference. Nature communications, 9(1), p.4273.
\item[\journal] Wijeratne, P.A., Young, A.L., Oxtoby, N.P., \textbf{Marinescu, R.V.}, Firth, N.C., Johnson, E.B., Mohan, A., Sampaio, C., Scahill, R.I., Tabrizi, S.J. and Alexander, D.C., 2018. An image‐based model of brain volume biomarker changes in Huntington's disease. Annals of clinical and translational neurology, 5(5), pp.570-582.
\item[\poster] Young, A.L., Scelsi, M.A., \textbf{Marinescu, R.V.}, Schott, J.M., Ourselin, S., Alexander, D.C. and Altmann, A., 2018. Genomewide Association Study Of Data-driven Alzheimer's Disease Subtypes. Alzheimer's \& Dementia: The Journal of the Alzheimer's Association, 14(7), pp.P1042-P1043.
\item[\poster] Garbarino, S., Lorenzi, M., Vinke, E., \textbf{Marinescu, R.V.}, Oxtoby, N.P., Eshaghi, A., Ikram, M.A., Niessen, W.J., Ciccarelli, O., Barkhof, F. and Vernooij, M.W., 2018. Mechanistic Profiles Of Neurodegeneration: A Study In Alzheimer’s Disease, Healthy Ageing And Primary Progressive Multiple Sclerosis. Alzheimer's \& Dementia: The Journal of the Alzheimer's Association, 14(7), pp.P1280-P1281.

\subsection*{2017}
\item[\poster] Young, A.L., \textbf{Marinescu, R.V.}, Yong, K., Firth, N.C., Oxtoby, N.P., Cash, D.M., Fox, N.C., Crutch, S.J., Rohrer, J.D., Schott, J.M. and Alexander, D.C., 2017. Characterising The Progression Of Alzheimer’s Disease Subtypes Using Subtype And Stage Inference (Sustain). Alzheimer's \& Dementia: The Journal of the Alzheimer's Association, 13(7), pp.P791-P792.
\item[\poster] Young, A.L., \textbf{Marinescu, R.V.}, Oxtoby, N.P., Bocchetta, M., Cash, D.M., Thomas, D.L., Dick, K.M., Cardoso, M.J., Ourselin, S., van Swieten, J.C. and Borroni, B., 2017. Multiple Distinct Atrophy Patterns Found In Genetic Frontotemporal Dementia Using Subtype And Stage Inference (Sustain). Alzheimer's \& Dementia: The Journal of the Alzheimer's Association, 13(7), pp.P453-P454.
\item[\poster] Primativo, S., \textbf{Marinescu, R.V.}, Firth, N.C., Yong, K., Shakespeare, T.J., Gonzalez, A.S., Carton, A.M., Lehmann, M., Slattery, C.F., Paterson, R.W. and Foulkes, A.J., 2017. Longitudinal Evaluation Of Neuropsychological And Neuroimaging Progression In Posterior Cortical Atrophy. Alzheimer's \& Dementia: The Journal of the Alzheimer's Association, 13(7), pp.P1382-P1383.
\item[\poster] Oxtoby, N.P., Young, A.L., \textbf{Marinescu, R.V.} and Alexander, D.C., 2017. Data-driven Models Of Disease Progression And Applications To Alzheimer’s Disease: Event-based Model And Differential Equation Models Of Biomarker Changes In ADNI. Alzheimer's \& Dementia: The Journal of the Alzheimer's Association, 13(7), pp.P1323-P1325.

\subsection*{2016}
\item[\poster] Firth, N.C., Brotherhood, E., Primativo, S., Young, A.L., \textbf{Marinescu, R.V.}, Oxtoby, N.P., Crutch, S.J. and Alexander, D.C., 2016. Data-driven Disease Progression Modelling Using Neuropsychological Tests: Posterior Cortical Atrophy Vs Alzheimer's Disease. Alzheimer's \& Dementia: The Journal of the Alzheimer's Association, 12(7), pp.P963-P964.

\subsection*{2015}
\item[\poster] Young, A.L., Oxtoby, N.P., Huang, J., \textbf{Marinescu, R.V.}, Daga, P., Cash, D.M., Fox, N.C., Ourselin, S., Schott, J.M., Alexander, D.C. and Alzheimers Disease Neuroimaging Initiative, 2015, June. Multiple orderings of events in disease progression. In International Conference on Information Processing in Medical Imaging (pp. 711-722). Springer, Cham.

\end{itemize}

\section*{Under review/In preparation}
\begin{itemize}
 \item \textbf{Marinescu, R.V.} et al, The Alzheimer's Disease Prediction Of Longitudinal Evolution (TADPOLE) Challenge: Results after 1 Year Follow-up, in preparation
\end{itemize}


\section*{Theses}
\begin{itemize}
 \item MEng thesis: On a new signature that quantifies topological structure in biological and economic networks. Supervisors: Natasa Przulj, Marek Sergot.
 \item PhD thesis: Modelling the Neuroanatomical Progression of Alzheimer's Disease and Posterior Cortical Atrophy. Supervisors: Daniel Alexander, Sebastian Crutch, Neil Oxtoby
\end{itemize}

\section*{Talks}
\begin{itemize}
 \item \emph{BrainPainter: A software for the visualisation of brain structures, biomarkers and associated pathological processes}, MICCAI MBIA workshop, 2019
 \item \emph{TADPOLE Challenge: Accurate Alzheimer's disease prediction through crowdsourced forecasting of future data}, MICCAI PRIME workshop, 2019
 \item \emph{Modelling the Neuroanatomical Progression of Alzheimer's Disease and Posterior Cortical Atrophy}, Athinoula A. Martinos Center, Cambridge MA, 2019
 \item \emph{A vertex clustering model for disease progression: application to cortical thickness images}. International Conference on Information Processing in Medical Imaging, 2017 (Erbsmann Prize Runner-up)
\end{itemize}

\section*{Review experience}
\begin{itemize}
 \item Information Processing in Medical Imaging (IPMI)
 \item Medical Image Computing and Computer Assisted Surgery (MICCAI)
 \item NeuroImage
 \item Alzheimer's and Dementia
 \item Neural Information Processing Systems (NeurIPS)
 \item Conference on Health, Inference, and Learning (CHIL)
\end{itemize}

\section*{News Coverage}
\begin{itemize}
 \item \url{https://www.alzforum.org/news/community-news/tadpole-challenge-seeks-best-predictors-alzheimers}
 \item \url{https://www.alzforum.org/news/community-news/tadpole-challenge-winners-forecast-ad-symptoms}
\end{itemize}


\section*{Software}
\begin{itemize}
 \item BrainPainter: \url{https://brainpainter.csail.mit.edu/}
\end{itemize}

\section*{About me}
\begin{itemize}
 \item Nationality: dual Romanian-British
 \item Languages spoken: Romanian (native), English (fluent), German (intermediate)
 \item Programming languages: Python, Java, C++, Haskell, Matlab, Prolog, Assembly x86
 \item Technical Experience with: Git, Vim, \LaTeX, OS programming, Compilers
\end{itemize}




%-------------------------------------------------------------------------------

\end{document}
